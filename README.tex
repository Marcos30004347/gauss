% Created 2022-02-05 sáb 17:12
% Intended LaTeX compiler: pdflatex
\documentclass[11pt]{article}
\usepackage[utf8]{inputenc}
\usepackage[T1]{fontenc}
\usepackage{graphicx}
\usepackage{grffile}
\usepackage{longtable}
\usepackage{wrapfig}
\usepackage{rotating}
\usepackage[normalem]{ulem}
\usepackage{amsmath}
\usepackage{textcomp}
\usepackage{amssymb}
\usepackage{capt-of}
\usepackage{hyperref}
\author{marcos}
\date{\today}
\title{Gauss}
\hypersetup{
 pdfauthor={marcos},
 pdftitle={Gauss},
 pdfkeywords={},
 pdfsubject={},
 pdfcreator={Emacs 27.1 (Org mode 9.3)}, 
 pdflang={English}}
\begin{document}

\maketitle
\tableofcontents



\section{Gauss}
\label{sec:org68e6c8b}
Gauss is a algebraic system written in C++

\subsection{Features:}
\label{sec:org970fc94}
\subsubsection{Creation of algebraic expressions}
\label{sec:org886083a}
\begin{verbatim}
#include <Algebra/Expressio.hpp>

expr x = symbol("x");
expr y = symbol("y");

expr a = 4*pow(x, 2) + 4*x + 13;
expr b = 6*pow(x,3)*y + 5*pow(y,2) + x*y;
\end{verbatim}

TODO:
\begin{itemize}
\item Add Integration Algorithms
\item Add Simplification of algebraic expressions
\item Add Type inference for algebraic expressions
\item Improve memory movement
\end{itemize}
\end{document}
